\documentclass[a4paper, 10pt]{article}
\usepackage[a4paper]{geometry}
\usepackage{mathtools}
\begin{document}

\title{Derive the HiTS-EQ direct fitting Equation}
\date{1.14.2015}
\author{Hsuan-Chun Lin \and Michael E. Harris \\
Case Western Reserve University}

\maketitle

\section{Things before PCR}

In the beginning, let us define some terms in this article.
Before PCR
The original substrate number for individual sequence in your sample without any binding $S_0$
The substrate number for individual sequence in the binding fraction $frac$ is $S_1$
The total sample counts before PCR is $S_t$ and the sample counts before PCR in the binding fraction $frac$ is $S_{t1}$.

\section{Things after PCR}
In PCR, the amplification fold for $S_t$ is $n_t$, and $n_{t1}$ for the binding fraction $S_{t1}$. The total counting number after deep sequencing is $D_{t}$ and $D_{t1}$.
Also the counting number for $S_0$ and $S_1$ after PCR are $D_0$ and $D_1$. Because different primer or PCR for each sequence may have bias, we set the error to adjust the amplification fold as $e_0$ and $e_1$.
For each sequence the amplification rate is $e_0*n_t$ and $e_1*n_{t1}$.

\section{Equation derivation}
The $E$ means enzyme concentration
The binding fraction could be written:
\[
f =\frac{S_0-S_1}{S_0}= \frac{E}{E+K_D}
\]
And we could write the Deep sequecing read as 
\[D_0 = S_0\times n_0e_0\]
\[D_1 = S_1\times n_1e_1\]
When we get our reads from High through-put sequencing, we can convert our reads to the original number of substrates as:
\[ S_0 = \frac{D_0}{n_0e_0} \] 
\[ S_1 = \frac{D_1}{n_1e_1} \]
Now we can rewrite the fitting equation as:
\[
f = \frac{ \frac{D_0}{n_0e_0} - \frac{D_1}{n_1e_1} }{\frac{D_0}{n_0e_0}} = \frac{E}{E+K_D}
\]
By rearranging the equation we can get the following equation:
\[
f = 1-(\frac{n_0}{n_1}\frac{e_0}{e_1}\frac{D_1}{D_0})
\]

We also know that 
\[
n_0 = \frac{D_t}{S_t}
\]
\[n_1 =\frac{D_{t1}}{S_{t1}}
\]
So we can get $n_1/n_0$ by:
\[
\frac{n_0}{n_1} = \frac{D_{t}S_t1}{D_t1S_{t}} = \frac{D_{t}}{D_t1}\times (1-frac)
\]
We also make the basic assumption that for each individual sequence, the error from sample preparation between samples is the same. which means
$$
\frac{e_0}{e_1} = 1 
$$
So our final equation for direct fitting is:
\[
f = 1 -( \frac{D_{t}}{D_t1}\times (1-frac) \times \frac{D_1}{D_0})= \frac{E}{E+K_D}
\]
Here $ \frac{D_{t1}}{D_t}$ is the ratio of total counts in sample with bounded fraction and the total counts in the unbound sample. The $\frac{D_1}{D_0}$ could be explained as mole fraction for each substrate.

\section{Experimental approach}
To make sure our basic postulate could establish, we need to focus on some details in the experiment. The most important one is PCR and primer design.
Our first hypothesis is that the errors from PCR and primers for a individual sequence in different binding fractions are the same ($e_0/e_1 = 1$). Which means we need to make sure the amplification in PCR should be well controlled. The real-time PCR could help us to check the amplification curve for each sample, primer and barcode. 
\end{document}